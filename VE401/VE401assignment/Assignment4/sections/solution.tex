\section*{Exercise 4.1}
\enum{
\item
Since
\[
    \textbf{AX}=(X_1,X_2)=\left(\begin{aligned}
        &a_{11}X_1+a_{12}X_2\\
        &a_{21}X_1+a_{22}X_2
    \end{aligned}\right).
\]
then
\begin{align*}
    E[\textbf{AX}]&=E\bigg[\left(\begin{aligned}
        &a_{11}X_1+a_{12}X_2\\
        &a_{21}X_1+a_{22}X_2
    \end{aligned}\right)\bigg]
    =\left(\begin{aligned}
        &E[a_{11}X_1+a_{12}X_2]\\
        &E[a_{21}X_1+a_{22}X_2]
    \end{aligned}\right)\\
    &=\left(\begin{aligned}
        &a_{11}E[X_1]+a_{12}E[X_2]\\
        &a_{21}E[X_1]+a_{22}E[X_2]
    \end{aligned}\right)
    =\textbf{A}\left(\begin{aligned}
        &E[X_1]\\
        &E[X_2]
    \end{aligned}\right)
    =\textbf{A}E[\textbf{X}].
\end{align*}

\item
\begin{align*}    
    \var(\textbf{AX})=\left(\begin{aligned}
        &\var(a_{11}X_1+a_{12}X_2) & \cov(a_{11}X_1+a_{12}X_2,a_{21}X_1+a_{22}X_2)\\
        &\cov(a_{21}X_1+a_{22}X_2,a_{11}X_1+a_{12}X_2) & \var(a_{21}X_1+a_{22}X_2)
    \end{aligned}\right)
\end{align*}
We denote that
\begin{align*}
    &t=\cov(X_1,X_2),\\
    &\cov(a_{21}X_1+a_{22}X_2,a_{11}X_1+a_{12}X_2)\\
    =&E[(a_{21}X_1+a_{22}X_2)(a_{11}X_1+a_{12}X_2)]-E[a_{21}X_1+a_{22}X_2]E[a_{11}X_1+a_{12}X_2]\\
    =&(a_{21}a_{12}+a_{22}a_{11})(E[X_1X_2]-E[X_1]E[X_2])+a_{11}a_{21}(E[X_1^2]-E[X_1]^2)+a_{22}a_{12}(E[X_2^2]-E[X_2]^2)\\
    =&(a_{21}a_{12}+a_{22}a_{11})\cov(X_1,X_2)+a_{11}a_{21}\var(X_1)+a_{22}a_{12}\var(X_2)\\
    =&(a_{21}a_{12}+a_{22}a_{11})t+a_{11}a_{21}\var(X_1)+a_{22}a_{12}\var(X_2),
\end{align*}
then
\begin{align*}
    \var(\textbf{AX})
    &=\left(\begin{aligned}
        &a_{11}^2\var(X_1)+a_{12}^2\var(X_2)+2a_{11}a_{12}t & \cov(a_{21}X_1+a_{22}X_2,a_{11}X_1+a_{12}X_2)\\
        &\cov(a_{21}X_1+a_{22}X_2,a_{11}X_1+a_{12}X_2) & a_{21}^2\var(X_1)+a_{22}^2\var(X_2)+2a_{21}a_{22}t
    \end{aligned}\right).
\end{align*}

Also,
\begin{align*}
    \textbf{A}\var(\textbf{X})&=\left(\begin{aligned}
        &a_{11}\var(X_1)+a_{12}\cov(X_1,X_2) & a_{11}\cov(X_1,X_2)+a_{12}\var(X_2)\\
        &a_{21}\var(X_1)+a_{22}\cov(X_1,X_2) & a_{21}\cov(X_1,X_2)+a_{22}\var(X_2)\\
    \end{aligned}\right).\\
    \textbf{A}^T&=\left(\begin{aligned}
        &a_{11} & a_{21}\\
        &a_{12} & a_{22}
    \end{aligned}\right).
\end{align*}
Hence, we know
\[
    \var(\textbf{AX})=\textbf{A}(\var(\textbf{X}))\textbf{A}^T.
\]

\item
Since $X_1$ and $X_2$ follow independent normal distributions, then $\varrho_{X}=0$.
\spl{
    f_X(x_1,x_2)=\frac{1}{2\pi\sigma_1\sigma_2}e^{-\frac{1}{2}\big[\frac{(x_1-\mu_1)^2}{\sigma_1^2}+\frac{(x_2-\mu_2)^2}{\sigma_2^2}\big]}.
}
\begin{align*}
    \Sigma_X=\left(\begin{aligned}
        &\sigma_1^2 & 0\\
        &0 & \sigma_2^2
    \end{aligned}\right), \quad
    \Sigma_X^{-1}=\frac{1}{\sigma_1^2\sigma_2^2}\left(\begin{aligned}
        &\sigma_2^2 & 0\\
        &0 & \sigma_1^2
    \end{aligned}\right)=\left(\begin{aligned}
        &\sigma_1^{-2} & 0\\
        &0 & \sigma_2^{-2}
    \end{aligned}\right),
\end{align*}

Hence,
\begin{align*}
    &\sqrt{\det\Sigma_X}=\sqrt{\sigma_1^2\sigma_2^2}=\sigma_1\sigma_2.\\
    &\Sigma_X^{-1}(x-\mu_X)=\left(\begin{aligned}
        &\sigma_1^{-2} & 0\\
        &0 & \sigma_2^{-2}
    \end{aligned}\right)
    \left(\begin{aligned}
        &x_1-\mu_1\\
        &x_2-\mu_2
    \end{aligned}\right)=
    \left(\begin{aligned}
        &\frac{x_1-\mu_1}{\sigma_1^2}\\
        &\frac{x_2-\mu_2}{\sigma_2^2}    
    \end{aligned}\right).
\end{align*}
Then,
\begin{align*}
    \bigg<x-\mu_X,\Sigma_X^{-1}(x-\mu_X)\bigg>=\frac{(x_1-\mu_1)^2}{\sigma_1^2}+\frac{(x_2-\mu_2)^2}{\sigma_2^2}.
\end{align*}

Thus,
\spl{
    f_X(x)=f_X(x_1,x_2)=\frac{1}{2\pi\sqrt{\det\Sigma_X}}e^{-\frac{1}{2}\big<x-\mu_X,\Sigma_X^{-1}(x-\mu_X)\big>}.
}

\item
Since \textbf{A} is invertible, then $X=A^{-1}Y$.
\begin{align*}
    f_Y(y)=f_X(x)|\det(A^{-1})|=f_X(A^{-1}y)|\det(A^{-1})|.
\end{align*}

\begin{align*}
    \det\Sigma_{Y}&=\det\left(\begin{aligned}
        &\var(Y_1) & \cov(Y_1,Y_2)\\
        &\cov(Y_2,Y_1) & \var(Y_2)
    \end{aligned}\right)=\sigma_{Y_1}^2\sigma_{Y_2}^2-\cov(Y_1,Y_2)^2\\
    &=(a_{11}^2\sigma_1^2+a_{12}^2\sigma_2^2)(a_{21}^2\sigma_1^2+a_{22}^2\sigma_2^2)-(a_{11}a_{21}\sigma_1^2+a_{22}a_{12}\sigma_2^2)^2\\
    &=(a_{11}a_{22}-a_{12}a_{21})^2\sigma_1^2\sigma_2^2\\
    &=\det(A)^2\det(\Sigma_X).
\end{align*}

Hence,
\begin{align*}
    &\sqrt{|\det\Sigma_Y|}=\sqrt{|\det(A)^2\det(\Sigma_X)|},\\
    &\sqrt{\det\Sigma_X}=\frac{1}{|\det(A)|}\sqrt{|\det\Sigma_Y|}.\\
    &\Sigma_Y^{-1}(y-\mu_Y)=\var(Y)^{-1}
    \left(\begin{aligned}
        &y_1-\mu_{Y_1}\\
        &y_2-\mu_{Y_2}
    \end{aligned}\right)\\
    =&(A^{T})^{-1}\Sigma_X^{-1}A^{-1}\left(\begin{aligned}
        &y_1-\mu_{Y_1}\\
        &y_2-\mu_{Y_2}
    \end{aligned}\right)
\end{align*}

Since
\begin{align*}
    \left(\begin{aligned}
        &y_1-\mu_{Y_1}\\
        &y_2-\mu_{Y_2}
    \end{aligned}\right)=
    \left(\begin{aligned}
        &a_{11}X_1+a_{12}X_2-(a_{11}\mu_1+a_{12}\mu_2)\\
        &a_{21}X_1+a_{22}X_2-(a_{21}\mu_1+a_{22}\mu_2)
    \end{aligned}\right)=
    A\left(\begin{aligned}
        &x_1-\mu_{_1}\\
        &x_2-\mu_{_2}
    \end{aligned}\right),
\end{align*}
then
\begin{align*}
    \Sigma_Y^{-1}(y-\mu_Y)=(A^{T})^{-1}\Sigma_X^{-1}(x-\mu_X).
\end{align*}

\begin{align*}
    \big<y-\mu_Y,\Sigma_Y^{-1}(y-\mu_Y)\big>&=\big<A(x-\mu_X),(A^T)^{-1}\Sigma_X^{-1}(x-\mu_X)\big>\\
    &=(x-\mu_X)^TA^T(A^T)^{-1}\Sigma_X^{-1}(x-\mu_X)\\
    &=\big<x-\mu_X,\Sigma_X^{-1}(x-\mu_X)\big>.
\end{align*}

Finally we plug in the terms,
\spl{
    f_Y(y)&=f_X(A^{-1}y)\det(A^{-1})\\
    &=\frac{|\det(A)|}{2\pi\sqrt{|\det\Sigma_Y|}}e^{-\frac{1}{2}\big<y-\mu_Y,\Sigma_Y^{-1}(y-\mu_Y)\big>}|\det(A^{-1})|\\
    &=\frac{1}{2\pi\sqrt{|\det\Sigma_Y|}}e^{-\frac{1}{2}\big<y-\mu_Y,\Sigma_Y^{-1}(y-\mu_Y)\big>}.
}

\item
We denote that $\varrho=\cov(Y_1,Y_2)/(\sigma_{Y_1}\sigma_{Y_2})$, then
\begin{align*}
    1-\varrho^2&=1-\frac{\cov^2(Y_1,Y_2)}{\sigma_{Y_1}^2\sigma_{Y_2}^2}\\
    &=\frac{\sigma_{Y_1}^2\sigma_{Y_2}^2-\cov^2(Y_1,Y_2)}{\sigma_{Y_1}^2\sigma_{Y_2}^2}\\
    &=\frac{\det\Sigma_Y}{\sigma_{Y_1}^2\sigma_{Y_2}^2}
\end{align*}

Hence,
\begin{align*}
    \sqrt{|\det\Sigma_Y|}=\sigma_{Y_1}\sigma_{Y_2}\sqrt{1-\varrho^2}.
\end{align*}

Also,
\begin{align*}
    &\big<y-\mu_Y,\Sigma_Y^{-1}(y-\mu_Y)\big>\\
    =&\frac{1}{1-\varrho^2}\frac{1}{\sigma_{Y_1}^2\sigma_{Y_2}^2}\big<y-\mu_Y,\Sigma_Y^*(y-\mu_Y)\big>\\
    =&\frac{1}{1-\varrho^2}\frac{1}{\sigma_{Y_2}^2\sigma_{Y_2}^2}(\sigma_{Y_1}^2(y_1-\mu_{Y_1})^2-2\cov(Y_1,Y_2)(y_1-\mu_{Y_1})(y_2-\mu_{Y_2})+\sigma_{Y_1}^2(y_2-\mu_{Y_2})^2)\\
    =&\frac{1}{1-\varrho^2}\big[\frac{(y_1-\mu_{Y_1})^2}{\sigma_{Y_1}^2}-2\varrho(\frac{y_1-\mu_{Y_1}}{\sigma_{Y_1}})(\frac{y_2-\mu_{Y_2}}{\sigma_{Y_2}})+\frac{(y_2-\mu_{Y_2})^2}{\sigma_2^2}\big]
\end{align*}
Hence,
\begin{align*}
    f_Y(y_1,y_2)=\frac{1}{2\pi\sigma_{Y_1}\sigma_{Y_2}\sqrt{1-\varrho^2}}e^{-\frac{1}{2(1-\varrho^2)}\big[\frac{(y_1-\mu_{Y_1})^2}{\sigma_{Y_1}^2}-2\varrho(\frac{y_1-\mu_{Y_1}}{\sigma_{Y_1}})(\frac{y_2-\mu_{Y_2}}{\sigma_{Y_2}})+\frac{(y_2-\mu_{Y_2})^2}{\sigma_2^2}\big]}.
\end{align*}

}

\section*{Exercise 4.2}
\begin{align*}
    E[Y]=E\bigg[\left(\begin{aligned}
        a_{11}X_1+a_{12}X_2\\
        a_{21}X_1+a_{22}X_2
    \end{aligned}\right)\bigg]=
    \left(\begin{aligned}
        a_{11}E[X_1]+a_{12}E[X_2]\\
        a_{21}E[X_1]+a_{22}E[X_2]
    \end{aligned}\right)=
    A\left(\begin{aligned}
        E[X_1]\\
        E[X_2]
    \end{aligned}\right)=AE[X].
\end{align*}

\begin{align*}
    \var(Y)&=\left(\begin{aligned}
        &\var(Y_1) & \cov(Y_1,Y_2)\\
        &\cov(Y_2,Y_1) & \var(Y_2)
    \end{aligned}\right)
    =\left(\begin{aligned}
        &(a_{11}^2+a_{12}^2)\sigma^2 & (a_{11}a_{21}+a_{22}a_{12})\sigma^2\\
        &(a_{11}a_{21}+a_{22}a_{12})\sigma^2 & (a_{21}^2+a_{22}^2)\sigma^2
    \end{aligned}\right)\\
    &=\sigma^2\left(\begin{aligned}
        &a_{11}^2+a_{12}^2 & a_{11}a_{21}+a_{22}a_{12}\\
        &a_{11}a_{21}+a_{22}a_{12} & a_{21}^2+a_{22}^2
    \end{aligned}\right).
\end{align*}

Since $A^T=A^{-1}$, then
\begin{align*}
    AA^T=I=\left(\begin{aligned}&1&0\\&0&1\end{aligned}\right)
        =\left(\begin{aligned}
        &a_{11}^2+a_{12}^2 & a_{11}a_{21}+a_{22}a_{12}\\
        &a_{11}a_{21}+a_{22}a_{12} & a_{21}^2+a_{22}^2
    \end{aligned}\right).
\end{align*}

Hence,
\begin{align*}
    \var(Y)=\sigma^2\left(\begin{aligned}&1&0\\&0&1\end{aligned}\right).
\end{align*}

\section*{Exercise 4.3}
Denote that $x=z_{\alpha_1}$ and $y=z_{\alpha_2}$. Then we obtain
\spl{
    &\Phi(-x)+\Phi(-y)=\alpha,\\
    &g(x,y):=\Phi(-x)+\Phi(-y)-\alpha=0.
}

Now we want to calculate the \textbf{conditional extreme} values of
\spl{
    f(x,y):=\frac{(x+y)\sigma}{\sqrt{n}}.
}

Thus we have the \textbf{partial differentiations} of $F(x,y,\lambda)=f(x,y)+\lambda g(x,y)$.

\[
    \left\{\begin{aligned}
        &F_x=\frac{\sigma}{\sqrt{n}}-\lambda f_N(-x)=0\\
        &F_y=\frac{\sigma}{\sqrt{n}}-\lambda f_N(-y)=0\\
        &F_\lambda=\Phi(-x)+\Phi(-y)-\alpha=0\\
    \end{aligned}\right. ,
\]

where $f_N(\cdot)$ is the density of a standard normal distribution.

We find that $x=y$, which means $z_{\alpha_1}=z_{\alpha_2}$. Also, $\alpha_1+\alpha_2=\alpha$.

Thus $\alpha_1=\alpha_2=\alpha/2$.

\section*{Exercise 4.4}
Since $(n-1)s^2/sigma_2$ follows a chi-squared distribution, then
\spl{
    1-\alpha&=P[\chi_{1-\alpha/2,n-1}^2\leq(n-1)s^2\sigma^2\leq\chi_{\alpha/2,n-1}^2]\\
    &=P\bigg[\frac{(n-1)s^2}{\chi_{\alpha/2,n-1}^2}\leq\sigma^2\leq\frac{(n-1)s^2}{\chi_{1-\alpha/2,n-1}^2}\bigg]\\
    &=P\bigg[\sqrt{\frac{(n-1)s^2}{\chi_{\alpha/2,n-1}^2}}\leq\sigma\leq\sqrt{\frac{(n-1)s^2}{\chi_{1-\alpha/2,n-1}^2}}\bigg].
}

Since $\alpha=0.05$, then
\spl{
    \chi_{0.025,50}^2&=71.42,\\
    \chi_{0.975,50}^2&=32.36.
}

Therefore, 
\spl{
    \sqrt{\frac{50\cdot0.37^2}{71.42}}\leq&\sigma\leq\sqrt{\frac{50\cdot0.37^2}{32.36}}\\
    0.31\leq&\sigma\leq0.46.
}

Hence, the 95\% two-sided confidence interval for $\sigma$ is [0.31,0.46].

\section*{Exercise 4.5}
\enum{
\item
We consider the exclusive situations. If all the samples are greater than $M$ or less than $M$, then $M$ will not fall between $X_{min}$ and $X_{max}$.

Since $F(M)=\frac{1}{2}$, which means the probability that a sample is less or greater than $M$ is both $\frac{1}{2}$.

Hence the probability that all the samples are greater than $M$ or less than $M$ is
\spl{
    \bigg(\frac{1}{2}\bigg)^n+\bigg(\frac{1}{2}\bigg)^n=\bigg(\frac{1}{2}\bigg)^{n-1}.
}

Therefore, the probability that $M$ falls between $X_{min}$ and $X_{max}$ is
\spl{
    P[X_{min}\leq M \leq X_{max}]=1-\bigg(\frac{1}{2}\bigg)^{n-1}.
}

\item 
For $P[X_{k+1}\leq M\leq X_{n-k}]$, we can also consider the exclusive situations. In this case, the probability that the samples $X_{k+1}$ to $X_{n-k}$ are greater than $M$ or less than $M$ is
\spl{
    \sum_{x=1}^k\binom{n}{x}\bigg(\frac{1}{2}\bigg)^{x}\bigg(\frac{1}{2}\bigg)^{n-x}+\sum_{x=1}^k\binom{n}{n-x}\bigg(\frac{1}{2}\bigg)^{x}\bigg(\frac{1}{2}\bigg)^{n-x}=\sum_{x=1}^k\binom{n}{x}\bigg(\frac{1}{2}\bigg)^{n-1}.
}

\spl{
    P[X_{k+1}\leq M\leq X_{n-k}]=1-\sum_{x=1}^k\binom{n}{x}\bigg(\frac{1}{2}\bigg)^{n-1}.
} 
}