\section*{Exercise 1.1}
The number of all the conditions that selecting 200 people from 2000 people is
\[
    N_1=\binom{2000}{120}.
\]
The number of the conditions that my friend and I are selected is equal to the number of conditions that selecting 118 people from 1998 people. That is
\[
    N_2=\binom{1998}{118}.
\]
Hence, the probability that my friend and I are selected is
\[
    P=\frac{\binom{1998}{118}}{\binom{2000}{120}}=\frac{1998!}{118!1880!}\div\frac{2000!}{120!1880!}\approx0.00357.
\]

\section*{Exercise 1.2}
\enum{
\item
We denote that $C=B\backslash A$, which means $A\cap C=\varnothing$.

Since $P[\bigcup_k A_k]=\sum_k P[A_k]$, then $P[B]=P[A\cup C]=P[A]+P[C]$.

Since $P[C]\geq 0$, then $P[B]\geq P[A]$.

Hence $P[A]\leq P[B]$.

\item
Since $A$ and $B$ are independent, then $P[A\cap B]=P[A]P[B]>0$.

Hence $A\cap B \neq \varnothing$.

Thus $A$ and $B$ are not mutually exclusive. 

\item
We denote that $C=A\cap B$, so that
\[
    (A\backslash C)\cap(C)\cap(B\backslash C)=\varnothing.
\]

Since $P[\bigcup_k A_k]=\sum_k P[A_k]$ for $\bigcap_k A_k=\varnothing$, then

\spl{
    P[A\cup B]&=P[(A\backslash C)\cup C\cup(B\backslash C)]\\
    &=P[A\backslash C]+P[C]+P[B\backslash C]\\
    &=P[A\backslash C]+P[C]+P[B\backslash C]\textbf{+P[C]-P[C]}\\
    &=(P[A\backslash C]+P[C])+(P[B\backslash C]+P[C])-P[C]\\
    &=P[A]+P[B]-P[C]\\
    &=P[A]+P[B]-P[A\cap B].
}
}

\section*{Exercise 1.3}
\enum{
\item
No.

Assume that the probability of "heads" for one toss is $p$, then
\spl{
    &P[\text{two heads}]=p^2=\frac{1}{3}\\
    &P[\text{no head}]=(1-p)^2=\frac{1}{3}\\
    &P[\text{one head}]=1-(1-p)^2-p^2=\frac{1}{3}\\
}
There's no such $p$ satisfying the equations.

\item 
No.

Assume that the probabilities of "heads" are respectively $p_1$ and $p_2$, then
\spl{
    &P[\text{two heads}]=p_1p_2=\frac{1}{3}\\
    &P[\text{no head}]=(1-p1)(1-p_2)=\frac{1}{3}\\
    &P[\text{one head}]=p_1(1-p_2)+p_2(1-p_1)=\frac{1}{3}\\
}
We obtain $
\left\{
\begin{aligned}
    p_1&=\frac{3+\sqrt{3}i}{6}\\ 
    p_2&=\frac{3-\sqrt{3}i}{6}\\ 
\end{aligned}
\right.
$

Hence there's no real root for $p_1$ and $p_2$.
}

\section*{Exercise 1.4}
\enum{
\item 
The probability that the selected participant has been asked the first question is $\frac{1}{2}$; the probability that the selected participant has \textbf{not} been asked the first question is $\frac{1}{2}$.

We denote "claim to see the barn" as "B", "has been asked the first question" as "A".

\[
    P=P[B|A]P[A]+P[B|\urcorner A]P[\urcorner A]=0.17\times0.5+0.03\times0.5=0.1
\]

\item
Since
\spl{
    P[A]&=0.5\neq0\\
    P[B|A]&=0.17,\\
}
then
\[
    P[B]=0.1\neq P[B|A].
\]
Hence seeing the barn is not independent of being asked the first question.
}

\section*{Exercise 1.5}
We suppose that the chips we can get are only from the market. Also, we suppose that the stolen chips are all sold on the market.

Denote that "being defective" as "D" and "being stolen" as "S", so that
\[
    P[S|D]=\frac{P[D|S]P[S]}{P[D]}.
\]
Since the stolen chips are stolen before inspection, then 
\begin{equation}\label{1}
    P[D|S]=0.5.
\end{equation}

For the probability that the chip we get was stolen,
\begin{equation}\label{2}
    P[S]=0.01.
\end{equation}

Finally,
\begin{equation}\label{3} 
    P[D]=P[D|\urcorner S]P[\urcorner S]+P[D|S]P[S]=0.05\times0.99+0.5\times0.01=0.0545.
\end{equation}

According to Results \ref{1}, \ref{2} and \ref{3}, we obtain
\[
    P[S|D]=\frac{0.5\times0.01}{0.0545}=\frac{10}{109}\approx0.0917.
\]

\section*{Exercise 1.6}
The prisoner is right.

Denote that "A", "B", "C" means A, B, C is going to die, respectively; "B*" means B is told not to die by the warden.

For example, B is told not to die. Then we obtain
\spl{
    P[A|B*]&=\frac{P[B*|A]P[A]}{P[B*|A]P[A]+P[B*|B]P[B]+P[B*|C]P[C]}\\ 
    &=\frac{\frac{1}{2}\times\frac{1}{3}}{\frac{1}{2}\times\frac{1}{3}+0\times\frac{1}{3}+1\times\frac{1}{3}}=\frac{1}{3}.
}
That is $P[C|B*]=1-P[A|B*]-P[B|B*]=1-\frac{1}{3}-0=\frac{2}{3}$.

In this case, the chance that A is going to die does not change, but the chance that C is going to die doubles.

\section*{Exercise 1.7}
\enum{
\item 
\[
m_X(t)=E[e^{tX}]=\sum_{k=1}^n e^{tx_k}\frac{1}{n}=\frac{1}{n}\sum_{k=1}{n}e^{tx_k},\quad t\in\mathbb{R}.
\]
\item 
\spl{
    E[X]=\frac{\dd m_X(t)}{\dd t}\bigg|_{t=0}=\frac{1}{n}\sum_{k=1}^n x_ke^{tx_k}\bigg|_{t=0}=\frac{1}{n}\sum_{k=1}^n x_k.
}
\spl{
    Var[X]&=E[X^2]-E[X]^2=\frac{\dd^2 m_X(t)}{\dd t^2}\bigg|_{t=0}-(\frac{\dd m_X(t)}{\dd t}\bigg|_{t=0})^2\\
    &=\frac{1}{n}\sum_{k=1}^n x_k^2 e^{tx_k}\bigg|_{t=0}-(\frac{1}{n}\sum_{k=1}^n x_k)^2\\
    &=\frac{1}{n}\sum_{k=1}^nx_k^2-(\frac{1}{n}\sum_{k=1}^n x_k)^2.
}
}

\section*{Exercise 1.8}
Since there exists some $\varepsilon$ such that $m_X(t)=m_Y(t)$ for all $t\in (-\varepsilon,\varepsilon)$, then their high order of derivatives exist and are the same. In other words,
\[
    \left\{
    \begin{aligned}
        &E[X]=\frac{\dd m_X(t)}{\dd t}\bigg|_{t=0}=\frac{\dd m_Y(t)}{\dd t}\bigg|_{t=0}=E[Y]\\
        &E[X^2]=\frac{\dd^2 m_X(t)}{\dd t^2}\bigg|_{t=0}=\frac{\dd^2 m_Y(t)}{\dd t^2}\bigg|_{t=0}=E[Y^2]\\
        &\vdots\\
        &E[X^{n}]=\frac{\dd^n m_X(t)}{\dd t^n}\bigg|_{t=0}=\frac{\dd^n m_Y(t)}{\dd t^n}\bigg|_{t=0}=E[Y^{n}]\\
    \end{aligned}
    \right.
\]

We also know that
\begin{equation}\label{eq1}
    \left\{
    \begin{aligned}
        &E[X]=\sum_{x=0}^nxf_X(x)\\
        &E[X^2]=\sum_{x=0}^nx^2f_X(x)\\
        &\vdots\\
        &E[X^n]=\sum_{x=0}^nx^nf_X(x)\\
    \end{aligned}
    \right.
\end{equation}

Similarly, for $(Y,f_Y)$,
\begin{equation}\label{eq2}
    \left\{
    \begin{aligned}
        &E[Y]=\sum_{x=0}^nxf_Y(x)\\
        &E[Y^2]=\sum_{x=0}^nx^2f_Y(x)\\
        &\vdots\\
        &E[Y^n]=\sum_{x=0}^nx^nf_Y(x)\\
    \end{aligned}
    \right.
\end{equation}

We also know that
\begin{equation}\label{eq3}
\begin{split}
    &f_X(0)+f_X(1)+\hdots+f_X(n)=1,\\
    &f_Y(0)+f_Y(1)+\hdots+f_Y(n)=1.
\end{split}
\end{equation}

To simplify the calculations, we denote that
\[
    \textbf{E}_{(n+1)\times1}=\left(
    \begin{aligned}
        &E[Y]\\
        &E[Y^2]\\
        &\vdots\\
        &E[Y^n]\\
        &1\\
    \end{aligned}
    \right)
    =
    \left(
    \begin{aligned}
        &E[X]\\
        &E[X^2]\\
        &\vdots\\
        &E[X^n]\\
        &1\\
    \end{aligned}
    \right),
\]
\[ 
    \textbf{A}_{(n+1)\times(n+1)}=\left(
    \begin{tabular}{cccc}
        0 & 1 & $\hdots$ & n\\
        $0^2$ & $1^2$ & $\hdots$ & $n^2$\\
        &$\vdots$&&\\
        $0^n$ & $1^n$ & $\hdots$ & $n^n$\\
        1 & 1 & $\hdots$ & 1\\
    \end{tabular}
    \right),
\]

where $\textbf{det(A)}\neq0$.

\[
    \textbf{X}_{(n+1)\times1}=\left(
    \begin{aligned}
        &f_X(0)\\
        &f_X(1)\\
        &\vdots\\
        &f_X(n-1)\\
        &f_X(n)\\
    \end{aligned}
    \right),\quad
    \textbf{Y}_{(n+1)\times1}=
    \left(
    \begin{aligned}
        &f_Y(0)\\
        &f_Y(1)\\
        &\vdots\\
        &f_Y(n-1)\\
        &f_Y(n)\\
    \end{aligned}
    \right)
\]

Hence, equations \ref{eq1}, \ref{eq2}, \ref{eq3} can be presented as
\begin{equation}\label{eq4}
\begin{split}
    &\textbf{A}\textbf{X}=\textbf{E}\\
    &\textbf{A}\textbf{Y}=\textbf{E} 
\end{split}
\end{equation}

We can tell that Eqs. \ref{eq4} are the same equation system. Since $\textbf{det(A)}\neq0$, we find that the equation system obtains only one solution.

Thus $\textbf{X}=\textbf{Y}$; that is
\begin{equation*}
    \left\{
    \begin{aligned}
        &f_X(0)=f_Y(0)\\
        &f_X(1)=f_Y(1)\\
        &\vdots\\
        &f_X(n)=f_Y(n)\\ 
    \end{aligned}
    \right.
\end{equation*}

In other words, $f_X(x)=f_Y(x)$ for $x=0,1,\hdots,n$.

\section*{Exercise 1.9}
\enum{
\item 
Applying the total probability formula, we obtain
\[
    P[Z=z]=\sum_{y=1}^zP[Z=z|Y=y]P[Y=y].
\]
Since X and Y are independent, then
\[
    P[Z=z|Y=y]=P[X=z-y|Y=y]=P[X=z-y].
\]

Thus
\[
    P[Z=z]=\sum_{y=1}^{z}P[X=z-y]P[Y=y]=\sum_{x+y=z}P[X=x]P[Y=y].
\]

\item 
For a geometric random variable, $P_g[X=x]=(1-p)^{x-1}p$.

For a pascal distribution with $r=2$, $$P_p[X=x]=\binom{x-1}{1}p^2(1-p)^{x-2}=(x-1)p^2(1-p)^{x-2}$$.

The sum of two independent geometric random variables is
\spl{
    P[Z=X+Y]&=P[Z=z]\\
    &=\sum_{x+y=z}P_g[X=x]P_g[Y=y]\\
    &=\sum_{x+y=z}(1-p)^{x-1}p(1-p)^{y-1}p\\
    &=\sum_{x+y=z}(1-p)^{x+y-2}p^2\\
    &=\sum_{x+y=z}(1-p)^{z-2}p^2
}
Since $x=1,\hdots,z-1$ and $y=1,\hdots,z-1$, then there are $z-1$ terms of $(1-p)^{z-2}p^2$.
\[
    P[Z=z]=(z-1)(1-p)^{z-2}p^2=P_p[X=x].
\]

Hence, the sum of two independent geometric random variables follows a Pascal distribution with $r = 2$.
}