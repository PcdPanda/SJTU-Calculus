\section*{Exercise 5.1}
We know that $\bar{x}=26.035$, $s=4.78476$. Here the null hypothesis is
\spl{
    H_0:\ \mu\leq25.
}

Hence,
\spl{
    &P[\overline{X}\geq\overline{x}\big|\mu\leq25]\leq P[\overline{X}\geq\overline{x}\big|\mu=25]\\
    =&P[\frac{\bar{X}-25}{s/\sqrt{n}}\geq\frac{26.035-25}{s/\sqrt{n}}]\\
    =&P[T_{19}\geq\frac{26.035-25}{4.78476}]\\
    =&P[T_{19}\geq0.97]\\
    \approx&0.17>0.05.
}

If we set $\alpha=0.05$, then $t_{0.05}(19)=1.729$.

We can reject $H_0$ if $t=0.97>t_{0.05}(19)$ but we fail, which means we say that we have no significant evidence to reject that $\mu\leq25$. Hence, we support the claim. The P-value is approximately 0.17.

\section*{Exercise 5.2}
\enum{
\item
We denote that $Z=\frac{\overline{X}-\mu_0}{\sigma/\sqrt{n}}$.
\spl{
    \alpha&=P[\text{reject }H_0\big|H_0 \text{ true}]\\
    &=P[\overline{X}\geq\overline{x}\big|\mu\leq4]\\
    &\leq P[\overline{X}\geq\overline{x}\big|\mu=4]\\
    &=P[Z\geq\frac{\overline{x}-4}{0.2/\sqrt{50}}]\\
    &=P[Z\geq z_{\alpha}]\\
    &\overset{!}{=}0.05.
}

Since $z_{0.05}=1.645$, then
\spl{
    \frac{\overline{x}-4}{0.2/\sqrt{50}}&=1.645\\
    \overline{x}&=4.047.
}

Thus the critical region is
\spl{
    \overline{X}\geq4.047.
}

\item
\spl{
    P&=P[\text{reject } H_0\big|H_1 \text{ true}]\\
    &=P[Z\geq z_{\alpha=0.05}\big|\mu\geq4.5]\\
    &\geq P[\frac{\overline{X}-4}{0.2/\sqrt{50}}\geq1.645\big|\mu=4.5]\\
    &=P[\overline{X}\geq4.047\big|\mu=4.5]\\
    &=1-\Phi(\frac{4.047-4.5}{0.2/\sqrt{50}})\\
    &=1.
}

Hence, the power of this test is 1.

\item
Since $z_{1-0.97}=-1.885$, then $z\leq-1.885$, which means
\spl{
    \frac{\overline{X}-4.5}{0.2/\sqrt{n}}=\frac{(z_{0.05}\times0.2/\sqrt{n}+4)-4.5}{0.2/\sqrt{n}}&\leq-1.885,\\
}
which gives $n\geq1.99$.

Hence, the sample size is at least 2.

\item
Since $\bar{x}=4.05>4.047$, we conclude that we can reject $H_0$ at $p<0.05$. 

A 95\% confidence interval for $\mu$ is $4.05\pm\frac{z_{0.025}\cdot0.2}{\sqrt{50}}=4.05\pm0.06$
}

\section*{Exercise 5.3}
\enum{
\item
\spl{
    n=8,\quad \sigma=4,\quad \mu_0=100.
}

Thus the normalized statistic is
\spl{
    z=\frac{\bar{x}-100}{4/\sqrt{8}}=1.556
}

Since $z_{0.05}=1.645<1.556$, then we conclude that we reject $H_0$ at $p<0.05$ .

\item
\spl{
    p&=P[\overline{X}\geq\bar{x}\big|\mu\leq100]\\
    &\leq P[Z\geq1.556]\\
    &=(0.0606+0.5904)/2\\
    &=0.06.
}

\item
\spl{
    \text{Power}&=P[\text{reject }H_0\big|\mu=105]=P[Z\geq z_{0.05}\big|\mu=105]\\
    &=P[\frac{\overline{X}-100}{4/\sqrt{8}}\geq z_{0.05}\big|\mu=105]\\
    &=P[\overline{X}\geq102.3\big|\mu=105]\\
    &=1-\Phi(\frac{102.3-105}{4/\sqrt{8}})\\
    &=0.9706.
}

\item
Since $z_{1-0.85}=-1.035$, then $z\leq-1.035$, which means
\spl{
    \frac{\overline{X}-105}{4/\sqrt{n}}=\frac{(z_{0.05}\times4/\sqrt{n}+100)-105}{4/\sqrt{n}}&\leq-1.035,\\
}
which gives $n\geq4.60$.

Hence, the sample size is at least 5.

\item
If we construct a 95\% lower confidence bound for $\mu$, which is
\spl{
    \mu\geq\overline{X}-\frac{z_{0.05}\cdot4}{\sqrt{8}}=102.2-2.3=99.9.
} 
Hence, we can say $\mu\geq99.9$ with the confidence of 95\%. It's equivalent to the test in part(i). The results are both rejecting $H_0$.
}

\section*{Exercise 5.4}
Denote that
\spl{
    G&=\sum_{k=1}^nX_k,\\
    K&=\frac{2}{\beta}\sum_{k=1}^nX_k.
}

\spl{
    F_K(y)&=P[K\leq y]=P[G\leq y\cdot\beta/2],\\
    f_K(y)&=F_K'(y)=\frac{\beta}{2}f_G(y\cdot\frac{\beta}{2}).
}

Then when $y>0$,
\spl{
    f_K(y)&=\frac{\beta}{2}\frac{1}{\Gamma(n)\beta^n}\bigg(y\cdot\frac{\beta}{2}\bigg)^{n-1}e^{-\frac{y\beta}{\beta\cdot2}}\\
    &=\bigg(\frac{\beta}{2}\bigg)^n\frac{1}{\Gamma(n)\beta^{n}}y^{n-1}e^{-\frac{y}{2}}\\
    &=\frac{1}{\Gamma(n)2^{n}}y^{n-1}e^{-\frac{y}{2}}\\
    &=\frac{1}{\Gamma(n'/2)2^{n'/2}}y^{n'/2-1}e^{-\frac{y}{2}}\\
    &=f_{\chi_{n'}^2}(y).
}

where $n'=2n$.

Hence, $K$ follows a chi-squared distribution with $2n$ degrees of freedom.

For $H_0:\ \beta=\beta_0$,
\spl{
    \chi_{1-\alpha/2,2n}^2 &\leq K\leq\chi_{\alpha/2,2n}^2\\
    \chi_{1-\alpha/2,2n}^2 &\leq \frac{2}{\beta}\sum_{k=1}^n \overline{X}_k\leq\chi_{\alpha/2,2n}^2\\
    \frac{\chi_{1-\alpha/2,2n}^2}{2\sum_{k=1}^n \overline{X}_k} &\leq \frac{1}{\beta}\leq\frac{\chi_{\alpha/2,2n}^2}{2\sum_{k=1}^n \overline{X_k}}\\ 
    \frac{2\sum_{k=1}^n \overline{X_k}}{\chi_{\alpha/2,2n}^2}&\leq\beta\leq\frac{2\sum_{k=1}^n \overline{X_k}}{\chi_{1-\alpha/2,2n}^2}.
}

Hence the critical region is 
\spl{
    \beta\leq\frac{2\sum_{k=1}^n \overline{X_k}}{\chi_{\alpha/2,2n}^2}\text{ or }\beta\geq\frac{2\sum_{k=1}^n \overline{X_k}}{\chi_{1-\alpha/2,2n}^2}.
}

For $H_0:\ \beta\leq\beta_0$,
\spl{
    K&\geq\chi_{\alpha,2n}^2\\
    \beta&\leq\frac{2\sum_{k=1}^n \overline{X_k}}{\chi_{\alpha,2n}^2}.
}

Hence the critical region is 
\spl{
    \beta\geq\frac{2\sum_{k=1}^n \overline{X_k}}{\chi_{\alpha,2n}^2}.
}