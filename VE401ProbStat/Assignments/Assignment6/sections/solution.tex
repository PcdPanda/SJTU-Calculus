\section*{Exercise 6.1}
\enum{
\item
There's no enough evidence that the success rate is greater for longer tears.

Large-sample test for differences in proportions will be used.

\spl{
    H_0:\ p_1-p_2=0
}
based on the statistic
\spl{
    Z=\frac{\hat{p_1}-\hat{p_2}}{\sqrt{\frac{\hat{p_1}(1-\hat{p_1})}{n_1}+\frac{\hat{p_2}(1-\hat{p_2})}{n_2}}}.
}

Here we obtain that
\spl{
    \hat{p_1}&=78\%,\\
    \hat{p_2}&=73\%,\\
    Z&=\frac{0.78-0.73}{\sqrt{\frac{0.78(1-0.78)}{18}+\frac{0.73(1-0.73)}{30}}}=0.39.
}

According to the standard normal distribution, $z_{0.05}=1.645$.
Hence $Z=0.39<z_{0.05}$, which means we have no sufficient evidence to support $p_1>p_2$.

\item

The standard deviation is
\spl{
    \sqrt{\frac{0.78(1-0.78)}{18}+\frac{0.73(1-0.73)}{30}}=0.127.
}

The 95\% confidence interval is
\spl{
    p_1-p_2>0+1.645\times0.127=0.209.
}
Thus
\spl{
    p_1-p_2>0.209.
}
}

\section*{Exercise 6.2}
\enum{
\item
The standard deviation is
\spl{
    \sqrt{\frac{\sigma_1^2}{n_1}+\frac{\sigma_2^2}{n_2}}=0.95.
}

\spl{
    \bar{x_1}-\bar{x_2}-z_{0.025}\sqrt{\frac{\sigma_1^2}{n_1}+\frac{\sigma_2^2}{n_2}}&\leq \mu_1-\mu_2\leq \bar{x_1}-\bar{x_2}+z_{0.025}\sqrt{\frac{\sigma_1^2}{n_1}+\frac{\sigma_2^2}{n_2}}\\
    -6-1.96\times0.95&\leq \mu_1-\mu_2\leq-6+1.96\times0.95\\
}
Hence, 
\spl{
    \mu_1-\mu_2=(-6.00\pm1.86)cm/s.    
}
\item
\spl{
    Z_0=\frac{\bar{x_1}-\bar{x_2}}{\sqrt{\frac{\sigma_1^2}{n_1}+\frac{\sigma_2^2}{n_2}}}=\frac{-6}{0.95}=-6.32<-1.96=-z_{0.025}.
}

Hence, we reject $H_0$ at a 5\% level of significance.

\item
We know that
\spl{
    \beta=1-p=0.10,\quad,\alpha=0.05,\quad \sigma=3cm/s,\quad\delta=14cm/s.
}

Hence,
\spl{
    n\approx\frac{2(z_{\alpha/2}+z_{\beta})^2\sigma^2}{\delta^2}=\frac{2(1.96+1.28)^2\times3^2}{14^2}=0.96.
}

}

\section*{Exercise 6.3}
\spl{
    H_0:\ \sigma_1^2=\sigma_2^2\quad H_1:\ |\sigma_1^2-\sigma_2^2|\geq0.5
}
Thus,
\spl{
    F_{n_1-1,n_2-1}&=F_{9,15}=\frac{s_1^2}{s_2^2}=\frac{4.7^2}{5.8^2}=0.66<f_{0.025,9,15}=3.1227\\
    F_{n_2-1,n_1-1}&=F_{15,9}=\frac{s_2^2}{s_1^2}=\frac{5.8^2}{4.7^2}=1.52<f_{0.025,15,9}=3.7694
}
We claim that $\sigma_1^2=\sigma_2^2$ at a 5\% level of significance.

Hence, there's no sufficient evidence to conclude that the two population variances differ by at least 0.5 grams per liter.

\section*{Exercise 6.4}
We set the hypotheses
\spl{
    H_0: \sigma_1^2=\sigma_2^2,\quad H_1: \sigma_1^2\neq\sigma_2^2
}
We know that
\spl{
    s_1^2=0.0015,\quad s_2^2=0.0120.
}

Then using $F$ test, we obtain
\spl{
    F_{n_1-1,n_2-1}=F_{9,9}=\frac{0.0015}{0.0120}=0.1234.
}
Hence,
\spl{
    F_{9,9}=0.1234<f_{0.025,9,9}&=4.0260\\
    F_{9,9}=0.1234<f_{0.975,9,9}&=0.2484\\
}


Thus, we claim $\sigma_1^2\neq\sigma_2^2$ at 5\% level of significance.

Since $0.1234\approx f_{0.9977,9,9}$, then the P-value is $0.0023\times2=0.0046$.

\section*{Exercise 6.5}
\enum{
\item
We set the hypothesis
\spl{
    H_0:\ \mu_1-\mu_2=0
}
We know form the table that
\spl{
    (\bar{x}_1-\bar{x}_2)=15.32-15.38=-0.06.
}
Thus we use pooled T-Test
\spl{
    S_p^2&=\frac{(9-1)\times0.0124+(9-1)\times0.0323}{9+9-2}=0.0223.\\
    T_{9+9-2}&=\frac{(\bar{X}_1-\bar{X}_2)-0}{\sqrt{0.0223(9^{-1}+9^{-1})}}=\frac{-0.06}{0.0704}=-0.852.\\
    |T_{16}|=0.852<t_{0.025,16}=2.1199.
}

Hence, we claim that $\mu_1=\mu_2$ at 5\% level of significance. The temperature does not affect the mean reading of concentration.
\item
We set the hypothesis
\spl{
    H_0:\ \mu_1-\mu_2=0
}
We know form the table that
\spl{
    (\bar{x}_1-\bar{x}_2)=0.430444-1.65844=-1.228.
}
Thus we use pooled T-Test
\spl{
    S_p^2&=\frac{(9-1)\times0.0420+(9-1)\times0.0660}{9+9-2}=0.054.\\
    T_{9+9-2}&=\frac{(\bar{X}_1-\bar{X}_2)-0}{\sqrt{0.054(9^{-1}+9^{-1})}}=\frac{-1.228}{0.110}=-11.21.\\
    |T_{16}|&=11.21>t_{0.025,16}=2.1199.
}

Hence, we claim that $\mu_1\neq\mu_2$ at 5\% level of significance. The temperature affects the mean reading of concentration.
}

\section*{Exercise 6.6}
\enum{
\item
\spl{
    H_0:\ \sigma_1^2=\sigma_2^2
}
We apply $F$ test.
\spl{
    F_{9,15}&=\frac{s_1^2}{s_2^2}=\frac{9}{7.29}=1.23\\
    f_{0.1,9,15}&=2.0862\\
    f_{0.9,9,15}&=0.4274\\
}

Since $F_{9,15}<f_{0.1,9,15}$ and $F_{9,15}>f_{0.9,9,15}$, then we claim that $\sigma_1^2=\sigma_2^2$ at 0.2 level of significance.
\item
\spl{
    s_p^2=\frac{(10-1)\times9+(16-1)\times7.9}{10+16-2}=8.3125.
}
\item
The statistic is 
\spl{
    T_{24}=\frac{\mu_1-\mu_2-(64.95-57.06)}{\sqrt{8.3125(10^{-1}+16^{-1})}}=\frac{\mu_1-\mu_2-7.89}{1.16}
}
Then, the 99\% confidence interval on $\mu_1-\mu_2$ is
\spl{
    \mu_1-\mu_2&=7.89\pm t_{0.005,24}\cdot1.16\\
    \mu_1-\mu_2&=7.89\pm3.25.
}
\item
We claim the there's a difference between $\mu_1$ and $\mu_2$ because we are 99\% sure that $\mu_1-\mu_2>4.64$.
}

\section*{Exercise 6.7}
We set the hypothesis
\spl{
    H_0:\ \mu_1=\mu_2 \quad H_1:\ \mu_1<\mu_2,
}
which is a one-sided hypothesis.

We know that
\spl{
    &\bar{x}_1=0.210,\quad s_1^2=0.001116.\\ 
    &\bar{x}_2=0.225,\quad s_2^2=0.001099.
}
Then,
\spl{
    T_{9+9-2}&=\frac{(\bar{X}_1-\bar{X}_2)-0}{\sqrt{s_p^2(9^{-1}+9^{-1})}}\\
    &=\frac{0.210-0.225}{\sqrt{(8\times0.001116+8\times0.001099)/16\times0.22}}\\
    &=\frac{-0.015}{0.0157}\\
    &=-0.955.
}

Since $t_{0.05,16}=1.7459$, then $T_{16}>-t_{0.05,16}$.

Hence, we claim that we have no sufficient evidence to support $\mu_1<\mu_2$ at 5\% level of significance.

\section*{Exercise 6.8}
\spl{
    H_0:\ \mu_1=\mu_2,\quad H_1: \mu_1>\mu_2
}
We denote $D=X_2-X_1$, so that
\spl{
    \overline{D}=-16.93, \quad s_D=33.7.
}
\spl{
    T_{15}=\frac{-16.93}{33.7/\sqrt{15}}=-1.945.
}

Since $t_{0.05,15}=1.7531$, then $T_{15}<-t_{0.05,15}$. Hence we reject $\mu_1=\mu_2$ and claim that $\mu_1>\mu_2$ at 5\% level of significance.

Using Wilcoxon Rank-Sum Test, we set
\spl{
    H_0:\ M_D=0\quad H_1:\ M_D<0.
}

\begin{table}[H]
    \centering
    \begin{tabular}{cc}\hline
        D & Rank\\\hline
        -4 & -1.5\\
        4 & 1.5\\
        9 & 3\\
        -11 & -4.5\\
        11 & 4.5\\
        14 & 6\\
        -15 & -7.5\\
        15 & 7.5\\
        -33 & -9.5\\
        33 & 9.5\\
        -37 & -11\\
        -46 & -12\\
        -48 & -13\\
        -56 & -14\\
        -90 & -15\\\hline
    \end{tabular}
\end{table}

Thus, $W_+=32$, $W_-=88$, $W=\min(W_-,W_+)=32$.
However, for $n=15$ at 5\% level of significance, we have a critical value $n=30$ for one-sided test. $W>30$.

Thus, we claim $M_1=M_2$ at 5\% level of significance.